\begin{eyenumerate}
   \item \PubItem[\with D. Fiorenza, G. Marchetti]{Hearts and towers in stable infinity-categories}
   {Journal of Homotopy and Related Structures (2019)}
   {We exploit the equivalence between t-structures and normal torsion theories on a stable ∞-category to show how a few classical topics in the theory of triangulated categories, i.e., the characterization of bounded t-structures in terms of their hearts, their associated cohomology functors, semiorthogonal decompositions, and the theory of tiltings, as well as the more recent notion of Bridgeland's slicings, are all particular instances of a single construction, namely, the tower of a morphism associated with a $J$-slicing of a stable ∞-category $\mathcal C$, where $J$ is a totally ordered set equipped with a monotone $\mathbb{Z}$-action.}
   \item \PubItem[\with E. Riehl]{Categorical notions of fibration}
   {Accepted by \emph{Expos. Math.}}
   {Fibrations over a category \textit{B}, introduced to category theory by Grothendieck, encode pseudo-functors $\textit{B°} \rightsquigarrow {\textbf{Cat}}$, while the special case of discrete fibrations encode presheaves $\textit{B°} \to \textbf{Set}$. A two-sided discrete variation encodes functors $\textit{B° x A} \to \textbf{Set}$, which are also known as profunctors from \textit{A} to \textit{B}. By work of Street, all of these fibration notions can be defined internally to an arbitrary 2-category or bicategory. While the two-sided discrete fibrations model profunctors internally to \textbf{Cat}, unexpectedly, the dual two-sided codiscrete cofibrations are necessary to model \textit{V}-profunctors internally to \textit{V}-\textbf{Cat}.}
   \item \PubItem[\with I. Di Liberti]{Homotopical Algebra is not concrete}
   {Journal of Homotopy and Related Structures (2017): 1-15.}
   {We generalize Freyd's well-known result that ``homotopy is not concrete'', offering a general method to show that under certain assumptions on a model category $\mathcal M$, its homotopy category $\text{ho}(\mathcal M)$ cannot be concrete. This result is part of an attempt to understand more deeply the relation between set theory and abstract homotopy theory.}
   \item \PubItem{Sober Ontic Structural Realism and Yoneda lemma}
   {presented at the Triennial conference of the “Società Italiana di Logica e Filosofia della Scienza”, Bologna}
   {A note on why the Yoneda lemma prevents to take too strong a position towards the non-existence of relata (\emph{radical} ontic structural realism posits that only relations exist).}
   \item \PubItem{This is the (co)end, my only (co)friend}
   {\arXiv{1501.02503v4} | submitted to LMS Lecture Note Series}
   {A survey of the most striking and useful applications of \emph{co/end calculus}. We put a considerable effort in making arguments and constructions rather explicit: after having given a series of preliminary definitions, we characterize co/ends as particular co/limits; then we derive a number of results directly from this characterization. The last sections discuss the most interesting examples where co/end calculus serves as a powerful abstract way to do explicit computations in diverse fields like Algebra, Algebraic Topology and Category Theory. The appendices serve to sketch a number of results in theories heavily relying on co/end calculus; the reader who dares to arrive at this point, being completely introduced to the mysteries of co/end fu, can regard basically every statement as a guided exercise.}
   \item \PubItem[\with D. Fiorenza]{t-structures are normal torsion theories}
   {Applied Categorical Structures 24.2 (2016): 181-208}
   {We characterize t-structures in stable ∞-categories as suitable quasicategorical factorization systems. More precisely we show that a t-structure $\texttt{t}$ on a stable ∞-category $\textbf{C}$ is equivalent to a normal torsion theory $\mathbb{F}$ on $\textbf{C}$, i.e. to a factorization system $\mathbb{F}=(\mathcal{E},\mathcal{M})$ where both classes satisfy the 3-for-2 cancellation property, and a certain compatibility with pullbacks/pushouts.}
\end{eyenumerate}
