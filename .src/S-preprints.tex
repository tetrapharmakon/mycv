\begin{eyenumerate}
   \item \PubItem[\with I. Di Liberti]{On the unicity of formal category theories}
   {\arXiv{1901.01594v1} | Submitted }
   {We prove an equivalence between cocomplete Yoneda structures and certain proarrow equipments on a 2-category $\mathcal K$. In order to do this, we recognize the presheaf construction of a cocomplete Yoneda structure as a relative, lax idempotent monad sending each admissible 1-cell $f :A \to B$ to an adjunction $\boldsymbol{P}_!f\dashv\boldsymbol{P}^*f$. Each cocomplete Yoneda structure on $\mathcal K$ arises in this way from a relative lax idempotent monad ``with enough adjoint 1-cells'', whose domain generates the ideal of admissibles, and the Kleisli category of such a monad equips its domain with proarrows. We call these structures ``yosegi''. Quite often, the presheaf construction associated to a yosegi generates an ambidextrous Yoneda structure; in such a setting there exists a fully formal version of Isbell duality.}
   \item \PubItem[\with I. Di Liberti]{Accessibility and presentability in 2-categories}
   {\arXiv{1804.08710v4} | Submitted to JPAA, January 2019}
   {We outline a definition of accessible and presentable objects in a 2-category $\mathcal K$ endowed with a Yoneda structure; this perspective suggests a unified treatment of many ``Gabriel-Ulmer like'' theorems (like the classical Gabriel-Ulmer representation for locally presentable categories, Giraud theorem, and Gabriel-Popescu theorem), asserting how presentable objects arise as reflections of generating ones. In a 2-category with a Yoneda structure, two non-equivalent definitions of presentability for $A\in\mathcal K$ can in principle be given: in the most interesting, it is generally false that all presheaf objects $\boldsymbol{P}A$ are presentable; this leads to the definition of a Gabriel-Ulmer structure, i.e. a Yoneda structure rich enough to concoct Gabriel-Ulmer duality and to make this asymmetry disappear. We end the paper with a roundup of examples, involving classical (set-based and enriched), low dimensional and higher dimensional category theory.}
   \item \PubItem{Localization theory for derivators}
   {\arXiv{1802.08193v1} | Submitted to TAC, March 2018}
   {We outline the theory of reflections for prederivators, derivators and stable derivators. In order to parallel the classical theory valid for categories, we outline how reflections can be equivalently described as categories of fractions, reflective factorization systems, and categories of algebras for idempotent monads. This is a further development of the theory of monads and factorization systems for derivators.}
   \item \PubItem[\with S. Virili]{Factorization systems on (stable) derivators}
   {\arXiv{1705.08565v3} | Submitted to JoA, June 2017}
   {We define triangulated factorization systems on triangulated categories, and prove that a suitable subclass thereof (the normal triangulated torsion theories) corresponds bijectively to t-structures on the same category. This result is then placed in the framework of derivators regarding a triangulated category as the base of a stable derivator. More generally, we define derivator factorization systems in the 2-category $\mathrm{PDer}$, describing them as algebras for a suitable strict 2-monad (this result is of independent interest), and prove that a similar characterization still holds true: for a stable derivator $\mathbb D$, a suitable class of derivator factorization systems (the normal derivator torsion theories) correspond bijectively with t-structures on the base $\mathbb{D}(\textbf{1})$ of the derivator. These two result can be regarded as the triangulated- and derivator- analogues, respectively, of the theorem that says that `t-structures are normal torsion theories' in the setting of stable ∞-categories, showing how the result remains true whatever the chosen model for stable homotopy theory is.}
   \item \PubItem[\with D. Fiorenza]{Recollements in stable ∞-categories}
   {\arXiv{1507.03913v2}}
   {We develop the theory of recollements in a stable ∞-categorical setting. In the axiomatization of Beilinson, Bernstein and Deligne, recollement situations provide a generalization of Grothendieck's ``six functors'' between derived categories. The adjointness relations between functors in a recollement $\textbf{D}^0, \textbf{D} , \textbf{D}^1$ induce a ``recoll\'ee'' t-structure $\texttt{t}_0\uplus\texttt{t}_1$ on $\textbf{D}$ , given t-structures $\texttt{t}_0,\texttt{t}_1$ on $\textbf{D}^0, \textbf{D}^1$. Such a classical result, well-known in the setting of triangulated categories, is recasted in the setting of stable ∞-categories and the properties of the associated (∞-categorical) factorization systems are investigated. In the geometric case of a stratified space, various recollements arise, which ``interact well'' with the combinatorics of the intersections of strata to give a well-defined, associative $\uplus$ operation. From this we deduce a generalized associative property for $n$-fold gluing $\texttt{t}_0\uplus\cdots\uplus \texttt{t}_n$, valid in any stable ∞-category.}
\end{eyenumerate}
